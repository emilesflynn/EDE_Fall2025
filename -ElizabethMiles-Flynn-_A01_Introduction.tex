% Options for packages loaded elsewhere
\PassOptionsToPackage{unicode}{hyperref}
\PassOptionsToPackage{hyphens}{url}
%
\documentclass[
]{article}
\usepackage{amsmath,amssymb}
\usepackage{iftex}
\ifPDFTeX
  \usepackage[T1]{fontenc}
  \usepackage[utf8]{inputenc}
  \usepackage{textcomp} % provide euro and other symbols
\else % if luatex or xetex
  \usepackage{unicode-math} % this also loads fontspec
  \defaultfontfeatures{Scale=MatchLowercase}
  \defaultfontfeatures[\rmfamily]{Ligatures=TeX,Scale=1}
\fi
\usepackage{lmodern}
\ifPDFTeX\else
  % xetex/luatex font selection
\fi
% Use upquote if available, for straight quotes in verbatim environments
\IfFileExists{upquote.sty}{\usepackage{upquote}}{}
\IfFileExists{microtype.sty}{% use microtype if available
  \usepackage[]{microtype}
  \UseMicrotypeSet[protrusion]{basicmath} % disable protrusion for tt fonts
}{}
\makeatletter
\@ifundefined{KOMAClassName}{% if non-KOMA class
  \IfFileExists{parskip.sty}{%
    \usepackage{parskip}
  }{% else
    \setlength{\parindent}{0pt}
    \setlength{\parskip}{6pt plus 2pt minus 1pt}}
}{% if KOMA class
  \KOMAoptions{parskip=half}}
\makeatother
\usepackage{xcolor}
\usepackage[margin=1in]{geometry}
\usepackage{graphicx}
\makeatletter
\def\maxwidth{\ifdim\Gin@nat@width>\linewidth\linewidth\else\Gin@nat@width\fi}
\def\maxheight{\ifdim\Gin@nat@height>\textheight\textheight\else\Gin@nat@height\fi}
\makeatother
% Scale images if necessary, so that they will not overflow the page
% margins by default, and it is still possible to overwrite the defaults
% using explicit options in \includegraphics[width, height, ...]{}
\setkeys{Gin}{width=\maxwidth,height=\maxheight,keepaspectratio}
% Set default figure placement to htbp
\makeatletter
\def\fps@figure{htbp}
\makeatother
\setlength{\emergencystretch}{3em} % prevent overfull lines
\providecommand{\tightlist}{%
  \setlength{\itemsep}{0pt}\setlength{\parskip}{0pt}}
\setcounter{secnumdepth}{-\maxdimen} % remove section numbering
\ifLuaTeX
  \usepackage{selnolig}  % disable illegal ligatures
\fi
\usepackage{bookmark}
\IfFileExists{xurl.sty}{\usepackage{xurl}}{} % add URL line breaks if available
\urlstyle{same}
\hypersetup{
  pdftitle={Assignment 1: Introduction},
  pdfauthor={Elizabeth Miles-Flynn},
  hidelinks,
  pdfcreator={LaTeX via pandoc}}

\title{Assignment 1: Introduction}
\author{Elizabeth Miles-Flynn}
\date{}

\begin{document}
\maketitle

\subsection{OVERVIEW}\label{overview}

This exercise accompanies the introductory material in Environmental
Data Analytics.

\subsection{Directions}\label{directions}

\begin{enumerate}
\def\labelenumi{\arabic{enumi}.}
\tightlist
\item
  Rename this file
  \texttt{\textless{}FirstLast\textgreater{}\_A01\_Introduction.Rmd}
  (replacing \texttt{\textless{}FirstLast\textgreater{}} with your first
  and last name).
\item
  Change ``Student Name'' on line 3 (above) with your name.
\item
  Work through the steps, \textbf{creating code and output} that fulfill
  each instruction.
\item
  Be sure to \textbf{answer the questions} in this assignment document.
\item
  When you have completed the assignment, \textbf{Knit} the text and
  code into a single PDF file.
\item
  After Knitting, submit the completed exercise (PDF file) to the
  appropriate assignment section on Canvas.
\end{enumerate}

\subsection{1) Discussion Questions}\label{discussion-questions}

Enter answers to the questions just below the \textgreater Answer:
prompt.

\begin{enumerate}
\def\labelenumi{\arabic{enumi}.}
\tightlist
\item
  What are your previous experiences with data analytics, R, and Git?
  Include both formal and informal training.
\end{enumerate}

\begin{quote}
Answer: This is my first time using Git. In my ecology undergraduate
classes, I would copy/paste R code from my professors to analyze our
field data with ANOVA, uploading excel tables, and editing graphs to fit
our lab report needs. I don't have any Python or formal coding
experience.
\end{quote}

\begin{enumerate}
\def\labelenumi{\arabic{enumi}.}
\setcounter{enumi}{1}
\tightlist
\item
  Are there any components of the course about which you feel confident?
\end{enumerate}

\begin{quote}
Answer: I feel comfortable with looking at online sources to help with
troubleshooting issues. Especially if there are any errors in my code, I
usually copy/paste the script and ask google search engine what steps or
lines I should fix.
\end{quote}

\begin{enumerate}
\def\labelenumi{\arabic{enumi}.}
\setcounter{enumi}{2}
\tightlist
\item
  Are there any components of the course about which you feel
  apprehensive?
\end{enumerate}

\begin{quote}
Answer: I want to refamiliarize myself with coding language. The only
ones I remember is that ``\textgreater{}'' is for writing code lines and
then ``\#'' is to help with notes that will not be coded. Overall, I
want to be confident with experimenting with R Studio and write my own
code from scratch that will be relevant to the modelling outcomes to
address project concerns.
\end{quote}

\subsection{2) GitHub}\label{github}

Provide a link below to your forked course repository in GitHub. Make
sure you have pulled all recent changes from the course repository and
that you have updated your course README file, committed those changes,
and pushed them to your GitHub account.

\begin{quote}
Answer: \url{https://github.com/emilesflynn/EDE_Fall2025}
\end{quote}

\subsection{3) Knitting}\label{knitting}

When you have completed this document, click the \texttt{knit}
button.This should produce a PDF copy of your markdown document. Submit
this PDF to Canvas

\end{document}
